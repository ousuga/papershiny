\documentclass[]{comunicaciones}
\labelpaper[year =2014 , month =12, volume =7, number =2 , firstpage =115 ]
%Cargar los paquetes necesarios aqu?
\usepackage{enumitem}  % format a list as to remove the spaces between list items
\usepackage[usenames,dvipsnames,svgnames,table]{xcolor} % para usar colores

% New for newcommand, proglang and code
\newcommand{\pkg}[1]{{\normalfont\fontseries{b}\selectfont #1}}
\let\proglang=\textsf
\let\code=\texttt

\begin{document}
\title[maintitle = Aplicaciones \pkg{shiny} para apoyar los procesos de aprendizaje de pruebas de hip?tesis (t?tulo tentativo a mejorar),
       secondtitle = T?tulo en ingl?s
       shorttitle = T?tulo corto]       

\begin{authors}
\author[firstname = Freddy,
surname = Hern?ndez Barajas,
%numberinstitution = 1,
affiliation = {Profesor asistente, Universidad Nacional de Colombia, Sede Medell\'in.},
email = fhernanb@unal.edu.co]
\author[firstname = Olga Cecilia,
surname = Usuga Manco,
%numberinstitution = 2,
affiliation = {Profesora asociada, Universidad de Antioquia, Medell\'in.},
email = olga.usuga@udea.edu.co]
\author[firstname = Santiago Humberto,
surname = Londo?o Restrepo,
%numberinstitution = 2,
affiliation = {Profesor de Ingenier\'ia , Corporaci\'on Universitaria Americana.},
email = slondono@coruniamericana.edu.co]
\end{authors}

\begin{mainabstract}
Resumen del art?culo.
\textcolor{red}{A cargo de Olga y Freddy.}
\keywords{Palabras claves}
\end{mainabstract}

\begin{secondaryabstract}
Resumen en ingl?s.
\textcolor{red}{A cargo de Olga y Freddy.}
\keywords{Palabras claves en ingl?s}
\end{secondaryabstract}

%%%%%%%%%%%%%%%%%%%%%%%%%%%%%%%%%%%%%%%%%%%%%%%%%%%%%%%%%%%%%%%%%%%%%%%%%%%%%%%%%%%%%%%%%%%%%%%%%%%%
\section{Introducci?n}
\textcolor{red}{A cargo de Olga.}


%%%%%%%%%%%%%%%%%%%%%%%%%%%%%%%%%%%%%%%%%%%%%%%%%%%%%%%%%%%%%%%%%%%%%%%%%%%%%%%%%%%%%%%%%%%%%%%%%%%%
\section{Pruebas de hip?tesis}
Bla bla bla.

%%%%%%%%%%%%%%%%%%%%%%%%%
\subsection{Prueba de hip?tesis para la varianza}
Suponga que se tiene una muestra aleatoria $x_1, x_2, \ldots, x_n$ proveniente de una poblaci?n normal. Se desea estudiar la hip?tesis nula $H_0: \sigma^2 = \sigma_0^2$ y se sospecha que la varianza $\sigma^2$ podr?a estar en alguna de las siguientes situaciones (hip?tesis alterna):
\begin{enumerate}[noitemsep, nolistsep]
	\item $H_1: \sigma^2 < \sigma_0^2$
	\item $H_1: \sigma^2 \neq \sigma_0^2$
	\item $H_1: \sigma^2 > \sigma_0^2$
\end{enumerate}
El estad?stico para realizar la prueba es:
$$\chi_0^2=\frac{(n-1) s^2}{\sigma_0^2},$$
donde $s$ desviaci?n est?ndar muestral. Bajo la suposici?n de que $H_0$ es verdadera, $\chi_0^2$ tiene distribuci?n $\chi^2$ con $n-1$ grados de libertad \cite{Montgomery03}.

%%%%%%%%%%%%%%%%%%%%%%%%%
\subsection{Prueba de hip?tesis para la media}
Suponga que se tiene una muestra aleatoria $x_1, x_2, \ldots, x_n$ proveniente de una poblaci?n normal. Se quiere estudiar la hip?tesis nula $H_0: \mu = \mu_0$ y se sospecha que la media $\mu$ podr?a estar en alguna de las siguientes situaciones (hip?tesis alterna):
\begin{enumerate}[noitemsep, nolistsep]
	 \item $H_1: \mu < \mu_0$
	 \item $H_1: \mu \neq \mu_0$
	 \item $H_1: \mu > \mu_0$
\end{enumerate}
El estad?stico para realizar la prueba es:
$$t_0=\frac{\bar{x} - \mu_0}{s/\sqrt{n}},$$
donde $\bar{x}$ y $s$ son la media y desviaci?n est?ndar muestrales respectivamente. Bajo la suposici?n de que $H_0$ es verdadera, el estad?stico $t_0$ tiene distribuci?n $t$-student con $n-1$ grados de libertad \cite{Walpole12}.

Si se d? el caso en que la muestra aleatoria no proviene de una poblaci?n normal pero se cumple que $n \geq 40$, entonces en virtud del Teorema del L?mite Central, el estad?stico para realizar la prueba es:
$$z_0=\frac{\bar{x} - \mu_0}{s/\sqrt{n}},$$
y en esta situaci?n el estad?stico $z_0$ tiene una distribuci?n $N(0, 1)$. 

En cualquiera de los casos, la hip?tesis nula $H_0$ se rechaza si el valor-P es menor que el nivel de significancia fijado previamente por el analista.

%%%%%%%%%%%%%%%%%%%%%%%%%
\subsection{Prueba de hip?tesis para la proporcion}
\textcolor{red}{A cargo de Olga.}

%%%%%%%%%%%%%%%%%%%%%%%%%
\subsection{Prueba de hip?tesis para el cociente de varianzas}\label{phvars}
Suponga que se tienen dos muestras aleatorias que provienen de poblaciones normales as?:
\begin{itemize}[noitemsep, nolistsep]
	\item $n_1$ observaciones $x_{11}, x_{12}, \ldots, x_{1,n1}$ de una poblaci?n I con  varianza $\sigma^2_1$, 
	\item $n_2$ observaciones $x_{21}, x_{22}, \ldots, x_{2,n2}$ de una poblaci?n II con varianza $\sigma^2_2$,
	\item ambas muestras son independientes entre s?.
\end{itemize}
Se quiere estudiar la hip?tesis nula $H_0: \sigma_1^2 / \sigma_2^2 = 1$ y se sospecha que el cociente de varianzas $\sigma_1^2 / \sigma_2^2$ podr?a estar en alguna de las siguientes situaciones (hip?tesis alterna):
\begin{enumerate}[noitemsep, nolistsep]
	\item $H_1: \sigma_1^2 / \sigma_2^2 < 1$
	\item $H_1: \sigma_1^2 / \sigma_2^2 \neq 1$
	\item $H_1: \sigma_1^2 / \sigma_2^2 > 1$
\end{enumerate}
El estad?stico para realizar la prueba es:
$$f_0=\frac{s_1^2}{s_1^2},$$

donde $s_1^2$ y $s_2^2$ son las varianzas muestrales de las poblaciones I y II respectivamente. El estad?stico $f_0$, bajo la suposici?n de que $H_0$ es verdadera, tiene distribuci?n $f$ con $n_1-1$ grados de libertad en el numerador y $n_2-1$ grados de libertad en el denominador \cite{Devore16}.

En esta prueba, al no rechazar la hip?tesis nula $H_0$, se concluye que $\sigma_1^2 / \sigma_2^2 = 1$ lo que implica en t?rminos pr?cticos que $\sigma_1^2 = \sigma_2^2$, es decir que las varianzas poblacionales se pueden considerar iguales.

%%%%%%%%%%%%%%%%%%%%%%%%%
\subsection{Prueba de hip?tesis para la diferencia de medias}
Suponga que se tienen dos muestras aleatorias que provienen de poblaciones normales as?:
\begin{itemize}[noitemsep, nolistsep]
	\item $n_1$ observaciones $x_{11}, x_{12}, \ldots, x_{1,n1}$ de una poblaci?n I con media $\mu_1$ y varianza $\sigma^2_1$,
	\item $n_2$ observaciones $x_{21}, x_{22}, \ldots, x_{2,n2}$ de una poblaci?n II con media $\mu_2$ y varianza $\sigma^2_2$,
	\item ambas muestras son independientes entre s?.
\end{itemize}
Se quiere estudiar la hip?tesis nula $H_0: \mu_1 - \mu_2 = \delta_0$ y se sospecha que la diferencia de medias $\mu_1 - \mu_2$ podr?a estar en alguna de las siguientes situaciones (hip?tesis alterna):
\begin{enumerate}[noitemsep, nolistsep]
	\item $H_1: \mu_1 - \mu_2 < \delta_0$
	\item $H_1: \mu_1 - \mu_2 \neq \delta_0$
	\item $H_1: \mu_1 - \mu_2 > \delta_0$
\end{enumerate}
Para realizar esta prueba de hip?tesis se deben diferenciar dos casos, uno en el que las varianzas son iguales y otro caso en el que las varianzas son diferentes, esto se puede chequear utilizando la prueba descrita en la secci?n \ref{phvars} del presente art?culo. Para cada uno de los casos descritos hay un estad?stico de prueba y una distribuci?n del estad?stico, a continuaci?n se presentan los dos casos en detalle.
\subsubsection{Caso 1: varianzas poblacionales iguales $\sigma_1^2 = \sigma_2^2$}
En este caso el estad?stico para realizar la prueba es:
$$t_0=\frac{\bar{x}_1 - \bar{x}_2 - \delta_0}{S_p \sqrt{\frac{1}{n_1} + \frac{1}{n_2}}},$$
donde $\bar{x}_1$ y $\bar{x}_2$ son las medias muestrales de las poblaciones I y II respectivamente, la cantidad $S_p^2$ es una varianza combinada y se calcula como:
$$S_p^2=\frac{(n_1-1)s_1^2+(n_2-1)s_2^2}{n_1+n_2-2},$$
donde $s_1^2$ y $s_2^2$ son las varianzas muestrales de las poblaciones I y II respectivamente.

En este caso el estad?stico $t_0$, bajo la suposici?n de que $H_0$ es verdadera, tiene distribuci?n $t$-student con $n_1+n_2-2$ grados de libertad \cite{Walpole12}.

\subsubsection{Caso 2: varianzas poblacionales diferentes $\sigma_1^2 \neq \sigma_2^2$}
En este caso el estad?stico para realizar la prueba es 
$$t_0=\frac{\bar{x}_1 - \bar{x}_2 - \delta_0}{\sqrt{\frac{s_1^2}{n_1} + \frac{s_2^2}{n_2}}}$$
En este caso el estad?stico $t_0$, bajo la suposici?n de que $H_0$ es verdadera, tiene distribuci?n $t$-student con $v$ grados de libertad \cite{Walpole12}, en donde $v$ se calcula como:
$$v=\frac{ \left( \frac{s_1^2}{n_1} + \frac{s_2^2}{n_2} \right)^2 }{ \frac{(s_1^2/n_1)^2}{n_1-1} + \frac{(s_2^2/n_2)^2}{n_2-1}}$$

%%%%%%%%%%%%%%%%%%%%%%%%%
\subsection{Prueba de hip?tesis para la diferencia de proporciones}
\textcolor{orange}{A cargo de Olga.}
Suponga que se tienen 

%%%%%%%%%%%%%%%%%%%%%%%%%%%%%%%%%%%%%%%%%%%%%%%%%%%%%%%%%%%%%%%%%%%%%%%%%%%%%%%%%%%%%%%%%%%%%%%%%%%%
\section{Paquete \pkg{shiny}}
\textcolor{red}{A cargo de Santiago.}

La idea es que aqu? se hable de los or?genes del paquete shiny, las partes de una aplicaci?n y otras generalidades. La idea es que el lector tenga un panorama muy general de una aplicaci?n, no que aprenda a construir una.

Se puede aqu? referenciar la p?gina de shiny en la cual se muestran tutoriales y la galeria:

https://shiny.rstudio.com/tutorial/

https://shiny.rstudio.com/gallery/

%%%%%%%%%%%%%%%%%%%%%%%%%%%%%%%%%%%%%%%%%%%%%%%%%%%%%%%%%%%%%%%%%%%%%%%%%%%%%%%%%%%%%%%%%%%%%%%%%%%%
\section{Aplicaciones \pkg{shiny} creadas}
\textcolor{red}{A cargo de Santiago.}

En esta secci?n se debe explicar el sitio donde est?n disponibles las aplicaciones, se podr?a colocar una imagen de una de las aplicaciones y explicar con detalle todas las partes y los cuidados que se deben tener. Podr?a ser una de las app de dos poblaciones, varianzas o medias.

Se debe explicar que la aplicaci?n puede recibir varios tipos de bases de datos (.csv, con espacios, etc).



%%%%%%%%%%%%%%%%%%%%%%%%%%%%%%%%%%%%%%%%%%%%%%%%%%%%%%%%%%%%%%%%%%%%%%%%%%%%%%%%%%%%%%%%%%%%%%%%%%%%
\section{Material de apoyo para los docentes}

\textcolor{red}{A cargo de Olga y Freddy.}

%%%%%%%%%%%%%%%%%%%%%%%%%%%%%%%%%%%%%%%%%%%%%%%%%%%%%%%%%%%%%%%%%%%%%%%%%%%%%%%%%%%%%%%%%%%%%%%%%%%%
\section{Conclusiones}

\begin{flushright}
\textbf{Recibido: }\\
\textbf{Aceptado: }
\end{flushright}

% Para la lista de referencias bibliogr?ficas, debe crear un archivo de extension bibtex y usar la siguiente l?nea
\nocite{*}
\references{Bibliografia}


\end{document}